\documentclass[a4paper,11pt]{article}
\usepackage[utf8]{inputenc}
\usepackage[english, russian]{babel}
\usepackage{amsfonts}
\usepackage{amsmath}
\usepackage{graphicx}
\begin{document}
\section{ Приветствие}
\label{sec:greetings}
Здравствуйте!
\section{Ссылка}
Я сделал приветствие в разделе~\ref{sec:greetings}
\begin{figure}[!h]
	\centering
		\includegraphics[width=0.5\textwidth]{pic01.png}
	\caption{Cсылка на раздел}
	\label{fig:ref}
\end{figure}
На рисунке ~\ref{fig:ref} показано, как ссылать на раздел

\begin{equation}
\label {eq:solve}
x^2 - 5 x + 6 = 0
\end{equation}
\begin{equation}
x_1 = \frac{5 + \sqrt{25 - 4 \times 6}}{2} = 3
\end{equation}
\begin{equation}
x_2 = \frac{5 - \sqrt{25 - 4 \times 6}}{2} = 2
\end{equation}

Таким образом, мы решили уравнение ~\eqref{eq:solve}

\section{Рождение пар внешнем полем}
В \cite{Sauter} было впервые предложено выражение для критического поля $E(s)$\ldots

\begin{thebibliography}{99}
\bibitem{Sauter} F.~Sauter, Z. Phys. \textbf{69}, 742; \textbf{73}, 547 (1931)
\bibitem{Yanovsky} V.~Yanovsky, V.~Chvykov, G.~Kalinchenko, \textit{et al}, Opt. Express, \textbf{16}, 2109 (2008).
\bibitem{ELI} \textit{Extreme Light Infrastructure: Report on the GrandChallenges Meeting}, edited by G.~Korn, P.~Antici (Paris, 2009).
\end{thebibliography}



\end{document}